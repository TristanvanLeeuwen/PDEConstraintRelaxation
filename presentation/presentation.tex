\documentclass{beamer}
\usetheme{Madrid}
\usepackage{amsmath, amssymb, graphicx}
\title[Constraint-Relaxation in PDE Inverse Problems]{An Analysis of Constraint-Relaxation in PDE-Based Inverse Problems}
\author{Tristan van Leeuwen \\ Utrecht University and CWI}
\date{International Workshop on Operator Learning and its Applications}

\begin{document}

\begin{frame}
  \titlepage
\end{frame}

\begin{frame}{Outline}
\begin{enumerate}
  \item Motivation and Examples
  \item Classical Approach
  \item Relaxed Formulation
  \item Main Result 1: Reduced Formulation
  \item Main Result 2: Limiting Cases
  \item Case Study: 1D Helmholtz
  \item Case Study: 3D Helmholtz
  \item Wrap-up
\end{enumerate}
\end{frame}

\begin{frame}{Motivation}
Inverse problems aim to estimate parameters in PDEs from observations.
\begin{itemize}
  \item Ubiquitous in science and engineering
  \item Typically formulated as PDE-constrained optimization
  \item Challenge: non-linearity, ill-posedness
\end{itemize}
\end{frame}

\begin{frame}{Examples}
Common applications:
\begin{itemize}
  \item Seismic imaging
  \item Medical imaging (e.g., MRI, EIT)
  \item Data assimilation in geosciences
\end{itemize}
%\includegraphics[width=0.8\textwidth]{example-placeholder}
\end{frame}

\begin{frame}{Classical Formulation}
Constrained optimization problem:
\begin{equation*}
\min_c \frac{1}{2} \sum_{i,j} |P_i(u_j) - d_{ij}|^2 \quad \text{s.t. } A_c(u_j, \varphi) = P_j(\varphi)
\end{equation*}
\begin{itemize}
  \item "Hard" PDE constraint
  \item Highly non-linear
\end{itemize}
\end{frame}

\begin{frame}{Motivation for Relaxation}
\begin{itemize}
  \item Improve optimization landscape
  \item Mitigate sensitivity to initialization
  \item Enlarge search space to parameter + state
\end{itemize}
\end{frame}

\begin{frame}{Relaxed Formulation}
Joint optimization problem:
\begin{equation*}
\min_{c, q} \frac{1}{2} \sum_{i,j} |P_i(u_j) - d_{ij}|^2 + \frac{\rho}{2} \sum_j \|q_j\|_U^2
\end{equation*}
Subject to:
\begin{equation*}
A_c(u_j, \varphi) = (P_j + R^{-1}q_j)(\varphi)
\end{equation*}
\end{frame}

\begin{frame}{Computational Challenges}
\begin{itemize}
  \item Optimization over large space $C \times U^n$
  \item Infeasible in practice
  \item Need reduced formulation
\end{itemize}
\end{frame}

\begin{frame}{Representer Theorem}
\textbf{Theorem 1:} Minimizers $q_j$ lie in span of $\{w_k\}$:
\begin{equation*}
q_j = \sum_{k=1}^n \alpha_{jk} w_k
\end{equation*}
with $A_c(\varphi, w_k) = P_k(\varphi)$
\end{frame}

\begin{frame}{Reduced Formulation}
\textbf{Corollary 1:}
\begin{equation*}
J(c) = \frac{1}{2} \sum_{j=1}^n \|e_j(c)\|_{(I + \rho^{-1} G(c))^{-1}}^2
\end{equation*}
This is an unconstrained optimization over $c$ only.
\end{frame}

\begin{frame}{Metric Definition}
\begin{itemize}
  \item $G(c)$ is Gram matrix from adjoint states:
\begin{equation*}
G_{ij}(c) = \langle w_i, w_j \rangle_U
\end{equation*}
  \item Can sometimes be computed from data
\end{itemize}
\end{frame}

\begin{frame}{Limiting Case $\rho \to \infty$}
\begin{equation*}
J(c) \to \frac{1}{2} \sum_{i,j} |P_i(u_j) - d_{ij}|^2
\end{equation*}
\begin{itemize}
  \item Recovers classical formulation
\end{itemize}
\end{frame}

\begin{frame}{Limiting Case $\rho \to 0$}
\begin{equation*}
J(c) \to \frac{\rho}{2} \sum_j \|e_j\|_{G(c)^{-1}}^2
\end{equation*}
\begin{itemize}
  \item Residuals weighted by inverse Gram matrix
  \item Convex under certain conditions
\end{itemize}
\end{frame}

\begin{frame}{Limiting Case Illustration}
%\includegraphics[width=0.9\textwidth]{figure1-placeholder}
\begin{itemize}
  \item Shows projections onto $\mathcal{P}_n$ and $\mathcal{W}_n$
\end{itemize}
\end{frame}

\begin{frame}{Projected Residuals}
\textbf{Theorem 2 and 3:} Interpret objective as projections
\begin{itemize}
  \item Solution residuals onto $\mathcal{P}_n$
  \item PDE residuals onto $\mathcal{W}_n$
\end{itemize}
\end{frame}

\begin{frame}{Case Study: 1D Helmholtz}
\begin{itemize}
  \item PDE: $-u'' - (k/c)^2 u = f$
  \item Measure: $d_{ij} = u_j(x_i)$
  \item Analytic solution available
\end{itemize}
\end{frame}

\begin{frame}{1D Helmholtz Results}
%\includegraphics[width=0.9\textwidth]{figure5-placeholder}
%\includegraphics[width=0.9\textwidth]{figure6-placeholder}
\end{frame}

\begin{frame}{Case Study: 3D Helmholtz}
\begin{itemize}
  \item Seismic inversion setup
  \item 124 sources/receivers
  \item Realistic starting models
\end{itemize}
\end{frame}

\begin{frame}{Reconstruction Results}
% \includegraphics[width=0.9\textwidth]{figure10-placeholder}
% \includegraphics[width=0.9\textwidth]{figure11-placeholder}
\end{frame}

\begin{frame}{Discussion}
\begin{itemize}
  \item Relaxation improves optimization landscape
  \item Bridges data-driven and PDE-based approaches
\end{itemize}
\end{frame}

\begin{frame}{Conclusion}
\begin{itemize}
  \item Presented unified framework for constraint-relaxation
  \item Shown reduced and limiting forms
  \item Case studies demonstrate advantages
\end{itemize}
\end{frame}

\begin{frame}{Acknowledgements}
\small
Thanks to collaborators, funding agencies, and the workshop organizers.
\end{frame}

\end{document}
